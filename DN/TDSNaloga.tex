\documentclass{article}

\usepackage[utf8]{inputenc}
\usepackage{amsmath}
\usepackage{mathtools}
\usepackage{graphicx}
\usepackage{subcaption}
\usepackage{caption}
\usepackage{float}
\usepackage{geometry}
\usepackage{physics}
\geometry{margin=1in}

\title{Domača naloga TDS}
\author{Andrej Kolar-Požun, 28172042}



\errorcontextlines 10000
\begin{document}
\pagenumbering{gobble}
\maketitle
\newpage
\pagenumbering{arabic}


Konstruiraj kanonične akcije in kote, ter poišči frekvence kot funkcije kanoničnih akcij za eliptični biljard. Poišči pogoje, da bo gibanje
periodično.


\section{Reševanje}
\subsection{Hamiltonjan v eliptičnih koordinatah}
Hamiltonjan za nalogo je(maso postavimo na 1):
\begin{equation*}
H = \frac{p^2}{2} + V(r) = \frac{\dot{x}^2 + \dot{y}^2}{2} + V(r)
\end{equation*}

Želimo ga izraziti s eliptičnimi koordinatami, ki se glasijo:
\begin{align*}
&x= f \ ch \mu \ \cos \nu \\
&y = f\  sh \mu \  \sin \nu \\
\end{align*}
Točki (f,0) in (-f,0) sta koordinati gorišč. a bom postavil na 1. 
Geometrijska predstava $\mu$ in $\nu$:
\begin{align*}
& \frac{x^2}{ ch^2 \mu} + \frac{y^2}{ sh^2 \mu} = 1 \\
& \frac{x^2}{ cos^2 \nu} - \frac{y^2}{ sin^2 \nu} = 1 \\
\end{align*}
Torej nam konstanten $\mu$ predstavlja elipso, $\nu$ pa hiperbolo.

\begin{figure}[H]
\centering
\begin{subfigure}{.7\textwidth}
\includegraphics[width=\linewidth]{grafi/prikaz.png}
\end{subfigure}
\end{figure}


S temi koordinatami izrazimo časovni odvod $(x,y)$:
\begin{align*}
&\dot{x} = sh(\mu) cos(\nu) \dot{\mu} - ch(\mu) sin(\nu) \dot{\nu} \\
&\dot{y} = ch(\mu) sin(\nu) \dot{\mu} + sh(\mu) cos(\nu) \dot{\nu} \\
&\dot{x}^2 + \dot{y}^2 =  (sh^2 \mu \ cos^2 \nu + ch^2 \mu \ sin^2 \nu) (\dot{\nu}^2 + \dot{\mu}^2) 
\end{align*}

Če to vstavimo nazaj v naš člen kinetične energije hamiltonjana in odvajamo, dobimo izraze za generalizirane momente v eliptičnih koordinatah:
\begin{align*}
&p_{\mu} =  (sh^2 \mu \ cos^2 \nu + ch^2 \mu \ sin^2 \nu) * \dot{\mu} = (ch^2 \mu - cos^2 \nu) \dot{\mu}= h^2 \dot{\mu} \\
&p_{\nu} =  (sh^2 \mu \ cos^2 \nu + ch^2 \mu \ sin^2 \nu) * \dot{\nu} = (ch^2 \mu - cos^2 \nu)\dot{\nu} = h^2 \dot{\nu}\\
\end{align*}
Kjer sem označil $h^2 = (ch^2 \mu - cos^2 \nu)$, saj se bo ta količina še večkrat pojavila.
Hamiltonjan tako postane
\begin{equation*}
H = \frac{p_{\mu}^2 + p_{\nu}^2}{2 h^2} + V(\mu)
\end{equation*}
Kjer je $V(\mu) =0$ Če $\mu < \mu_{max}$ in $V(\mu) = \infty$ sicer.
\subsection{Integrali gibanja}

Ker Hamiltonjan ni eksplicitno odvisen od časa je prvi integral gibanja energija. Torej 
\begin{equation*}
F_1 = \frac{p_{\mu}^2 + p_{\nu}^2}{2 h^2} = E
\end{equation*}

Saj je potencial v območju gibanja enak nič.

Drugi pa je (namig) zmnožek vrtilnih količin glede na prvo in drugo gorišče elipse torej:
\begin{align*}
&F_2 = ((x-1) p_{y} - y p_{x})((x+1) p_{y} - y p_{x}) = (x^2-1)p_y^2 + y^2 p_x^2 - 2yx p_y p_x  
\end{align*}
Ko zgornje količine izrazimo z eliptičnimi koordinatami dobimo:
\begin{align*}
&F_2 =  (ch^2(\mu) - cos^2(\nu))(\dot{\nu}^2 sh^2(\mu) - \dot{\mu}^2 sin^2(\nu)) =   h^2 (\dot{\nu}^2 sh^2(\mu) - \dot{\mu}^2 sin^2(\nu)) \\
&F_2 = \frac{1}{h^2} (p_{\nu}^2 sh^2 \mu - p_{\mu}^2 sin^2 \nu) = \Gamma
\end{align*}

S pomočjo konstant gibanja lahko izrazimo
\begin{align*}
&p_{\nu}^2 = 2E sin^2 \nu + \Gamma  \\
&p_{\mu}^2 = 2E sh^2 \mu - \Gamma
\end{align*}
Z uvedbo brezdimenzijske spremenljivke $k = \frac{\Gamma}{2E}$ izraz postane
\begin{align*}
&p_{\nu}^2 = 2E (sin^2 \nu + k)  \\
&p_{\mu}^2 = 2E (sh^2 \mu - k)
\end{align*}
\subsection{Kanonične akcije in koti}
\begin{equation*}
J_{\mu} = \frac{2}{2\pi} \int_{\mu_{min}}^{\mu_{max}} p_{\mu} d\mu
\end{equation*}
Tukaj je $\mu_{max}$ meja biljarda, kar je inverzni hiperbolični tangens razmerja male in velike polosi.
$\mu_{min}$ je odvisen od gibanja, ki nas zanima. Če si narišemo dinamiko(slike pridejo pozneje) vidimo, da imamo za pozitivne k nekako rotirajoče gibanje. Kavstika(envelopa trajektorij) je neka manjša elipsa, torej dobimo v tem primeru $\mu_{min}$ tako, da zahtevamo, da je $p_{\mu}=0$.
Dobimo, da je $\mu_{min} = arsh(\sqrt{k})$
Torej:
\begin{align*}
&J_{\mu} = \frac{\sqrt{2E}}{\pi} \int_{arsh\sqrt{k}}^{\mu_{max}} \sqrt{sh^2 \mu - k} d\mu 
\end{align*}
Akcija, ki ustreza kanonični koordinati $\nu$ je preprostejša saj so meje fiksne:
\begin{equation*}
J_{\nu} = \frac{4\sqrt{2E}}{2 \pi} \int_0^{\pi/2} \sqrt{sin^2 \nu + k} d\nu 
\end{equation*}

Za drugi način gibanja(negativni k), bi postopali podobno, le da so kavstike potem hiperbole.


Ustrezne frekvence dobimo kot odvod hamiltonjana po teh akcijah. Račun se zelo zakomplicira in ga tukaj ne bom izvajal(Rezultat lahko najdete v članku "Probability distributions in classical and quantum elliptic billiards", Vega and Cerda, 2001).
Gibanje je periodično, ko je razmerje obeh frekvenc racionalno.

\section{Grafi}

Parameter, ki določa tipe gibanja je k. Za pozitiven k imamo rotirajoče gibanje(kavstike so elipse), za negativne nekako skakajoče(kavstike so hiperbole).
Mejne vrednosti za k so -1 in $(b/f)^2$, kjer je b mala polos.

\begin{figure}[H]
\centering
\begin{subfigure}{.7\textwidth}
\includegraphics[width=\linewidth]{grafi/5.pdf}
\end{subfigure}
\end{figure}

\begin{figure}[H]
\centering
\begin{subfigure}{.7\textwidth}
\includegraphics[width=\linewidth]{grafi/6.pdf}
\end{subfigure}
\end{figure}

\begin{figure}[H]
\centering
\begin{subfigure}{.7\textwidth}
\includegraphics[width=\linewidth]{grafi/7.pdf}
\end{subfigure}
\end{figure}

\begin{figure}[H]
\centering
\begin{subfigure}{.7\textwidth}
\includegraphics[width=\linewidth]{grafi/8.pdf}
\end{subfigure}
\end{figure}

\begin{figure}[H]
\centering
\begin{subfigure}{.7\textwidth}
\includegraphics[width=\linewidth]{grafi/9.pdf}
\end{subfigure}
\end{figure}

\begin{figure}[H]
\centering
\begin{subfigure}{.7\textwidth}
\includegraphics[width=\linewidth]{grafi/1.pdf}
\end{subfigure}
\end{figure}

\begin{figure}[H]
\centering
\begin{subfigure}{.7\textwidth}
\includegraphics[width=\linewidth]{grafi/2.pdf}
\end{subfigure}
\end{figure}

\begin{figure}[H]
\centering
\begin{subfigure}{.7\textwidth}
\includegraphics[width=\linewidth]{grafi/3.pdf}
\end{subfigure}
\end{figure}

\begin{figure}[H]
\centering
\begin{subfigure}{.7\textwidth}
\includegraphics[width=\linewidth]{grafi/4.pdf}
\end{subfigure}
\end{figure}
\end{document}